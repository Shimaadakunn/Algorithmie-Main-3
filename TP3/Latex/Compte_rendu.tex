\documentclass[a4paper]{article}
\usepackage[top=2cm, bottom=2cm, left=2cm, right=2cm]{geometry}
\usepackage[utf8]{inputenc}
\usepackage{subcaption,graphicx}
\usepackage[T1]{fontenc}
\usepackage[french]{babel}

\title{Compte rendu TP4 Algorithmie}
\author{Léo Combaret}
\date{Avril 2022}

\begin{document}
    %espacement entre les lignes d'un tableau
\renewcommand{\arraystretch}{1.5}
%régler l'espacement entre les lignes
\newcommand{\HRule}{\rule{\linewidth}{0.5mm}}
\maketitle

\section{Arbre AVL}

Dans ce TP, nous voulions créer un arbre balisé.
Cependant pour pouvoir baliser un arbre, il faut d'abord qu'il soit équilibré.\\

L'idée de mon algorithmie était d'avoir une simple fonction \emph{add} qui ajoute le noeud dans un arbre puis rééquilibre l'arbre.
Dans cette fonction, après avoir placé le noeud à la bonne place dans l'arbre, elle appelle d'abord la fonction \emph{abrDeseq} qui 
met à jour l'indice de déséquilibre de tous les noeuds de l'arbre. \\

Ensuite, la fonction utilise la fonction \emph{abrRotate} qui va équilibrer le noeud déséquilibré.
Pour cela, elle fait appelle à la fonction \emph{findDeseqKey} qui retourne la clé du noeud à l'origine de ce déséquilibre.
Une fois ce noeud trouvé, on parcours l'abre jusqu'à ce noeud, puis on effectue une des quatres types de rotations selon la valeur des clés.\\\\
 - Rotation gauche : Le père a un indice de déséquilibre de 2 et son fils droit de 1\\
 - Rotation droite : Le père a un indice de déséquilibre de -2 et son fils gauche de -1\\
 - Rotation droite gauche : Le père a un indice de déséquilibre de 2 et son fils de -1\\
 - Rotation gauche droite :  Le père a un indice de déséquilibre de -2 et son fils de 1\\\\
Ainsi, l'arbre sera toujours équilibré pour pouvoir mettre les balises sur les feuilles avec un parcours infixe

 \end{document}
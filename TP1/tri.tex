\documentclass[a4paper]{article}
\usepackage[top=2cm, bottom=2cm, left=2cm, right=2cm]{geometry}
\usepackage[utf8]{inputenc}

\title{Compte rendu TP1 Algorithmie}
\author{Léo Combaret}
\date{Février 2022}

\begin{document}
    %espacement entre les lignes d'un tableau
\renewcommand{\arraystretch}{1.5}
%régler l'espacement entre les lignes
\newcommand{\HRule}{\rule{\linewidth}{0.5mm}}
\maketitle

\section{Réaliser un suivi à la trace}

Dans cette première section, nous voulons visualiser ce que notre fonction $partitionBis()$ effectue pour placer le pivot au bon endroit. \\
\subsection{}
Pour le tableau $I_{1}$ = [2,6,0,4,3,1,5] \\

 \begin{tabular}{ | c | c | c | c | c | c |}
   
   Tableau          & Pivot & j     & T[j]      & pospiv & comparaison effectuée\\
   \hline
   
   [2,6,0,4,3,1,5]  & 2     & 1     & 6         & 0      & 6 $!<=$ 2\\
   
   [2,6,0,4,3,1,5]  & 2     & 2     & 0         & 1      & 0 $<=$ 2\\
   
   [2,0,6,4,3,1,5]  & 2     & 2     & 0         & 1      & echange(6,0)\\
   
   [2,0,6,4,3,1,5]  & 2     & 3     & 4         & 1      & 4 $!<=$ 2 \\
   
   [2,0,6,4,3,1,5]  & 2     & 4     & 3         & 1      & 3 $!<=$ 2\\
   
   [2,0,6,4,3,1,5]  & 2     & 5     & 1         & 2      & 1 $<=$ 2\\
   
   [2,0,1,4,3,6,5]  & 2     & 5     & 1         & 2      & echange(6,1)\\
   
   [2,0,6,4,3,1,5]  & 2     & 6     & 5         & 2      & 5 $!<=$ 2\\
   
   [1,0,2,4,3,1,5]  & 2     & 6     & 5         & 2      & echange(2,1)\\
   
 \end{tabular}
\vspace*{5 mm}\\
Ainsi, nous pouovons déduire que pour cette instance, il y a 6 comparaisons et 3 échanges soit 6 affectations. (1 échange = 2 affectations)

\subsection{}
Pour le tableau $I_{2}$ = [7,6,5,4,3,2,1] \\

 \begin{tabular}{ | c | c | c | c | c | c |}
   
   Tableau          & Pivot & j     & T[j]      & pospiv & comparaison effectuée\\
   \hline
   
   [7,6,5,4,3,2,1]  & 7     & 1     & 6         & 1      & 6 $<=$ 2\\
   
   [7,6,5,4,3,2,1]  & 7     & 2     & 5         & 2      & 5 $<=$ 7\\
   
   [7,6,5,4,3,2,1]  & 7     & 3     & 4         & 3      & 4 $<=$ 7 \\
   
   [7,6,5,4,3,2,1]  & 7     & 4     & 3         & 4      & 3 $<=$ 7\\
   
   [7,6,5,4,3,2,1]  & 7     & 5     & 2         & 5      & 2 $<=$ 7\\
   
   [7,6,5,4,3,2,1]  & 7     & 6     & 1         & 6      & 1 $<=$ 7\\
   
   [1,6,5,4,3,2,7]  & 7     & 6     & 1         & 6      & echange(7,1)\\
   
 \end{tabular}
\vspace*{5 mm}\\
De même, ici nous obtenons 6 comparaisons et 2 affectations.

\subsection{}
Pour le tableau $I_{3}$ = [1,2,3,4,5,6,7] \\

 \begin{tabular}{ | c | c | c | c | c | c |}
   
   Tableau          & Pivot & j     & T[j]      & pospiv & comparaison effectuée\\
   \hline
   
   [1,2,3,4,5,6,7]  & 1     & 1     & 6         & 0      & 2 $!<=$ 1\\
   
   [1,2,3,4,5,6,7]  & 1     & 2     & 5         & 0      & 3 $!<=$ 1\\
   
   [1,2,3,4,5,6,7]  & 1     & 3     & 4         & 0      & 4 $!<=$ 1 \\
   
   [1,2,3,4,5,6,7]  & 1     & 4     & 3         & 0      & 5 $!<=$ 1\\
   
   [1,2,3,4,5,6,7]  & 1     & 5     & 2         & 0      & 6 $!<=$ 1\\
   
   [1,2,3,4,5,6,7]  & 1     & 6     & 1         & 0      & 7 $!<=$ 1\\
   
 \end{tabular}
\vspace*{5 mm}\\
Enfin, ici nous observons 6 comparaisons et 0 affectation.
\section{Démonstration de la technique $partitionBis$ et de sa compléxité}

\subsection{Démonstration}
La logique de la fonction $partitionBis$ est la suivante: 
\\
On va comparer le premier élément du tableau (le pivot) avec tous les autres éléments du tableau de gauche à droite.\\
Si l'élément avec lequel on le compare est plus grand, on continue l'incrémentation.\\
Si l'élément est plus petit ou égale, on va avancer la "future place" du pivot de 1.\\
Si cette future place est occupée par une valeur plus grande que le pivot, on l'échange avec la valeur plus petit que le pivot que l'on vient de comparer afin qu'il n'y ai que des éléments plus petit à gauche de la "future place" du pivot.\\
Ainsi on arrive, pour chaque appel de la fonction, de bien partitionner le tableau.


\subsection{Complexité}
On veut calculer le nombre moyen de comparaison $C_{N}$ pour trier avec $partitionBis$ un tableau à $N$ éléments.
On considère que $C_{0} = C_{1} = 1$.
Pour trier $N$ éléments on doit faire une partition. Cette partition va demander $N+1$ comparaisons.
Ensuite, on doit faire deux appels récurssifs. La compléxité de ces appels récurssifs dépend évidement de la position du pivot et donc du nombre d'éléments des deux sous-tableaux.\\
Vu que dans cette fonction $partitionBis$ le choix de notre pivot est fait de manière aléatoire car on prend le premier élément de notre tableau. On a donc une chance sur N que le pivot se trouve à chaque endroit du tableau.\\
\vspace*{1 mm}\\
On peut alors en déduire l'égalité suivante: \\
\begin{equation}
    C_{N} = (N) + \left( \frac{C_{0} + C_{N-1}}{N} \right) + \left( \frac{C_{1} + C_{N-2}}{N} \right) + \cdots + \left( \frac{C_{N-1} + C_{0}}{N} \right)
\end{equation}\\
A partir de cette relation, on peut calculer:
\begin{equation}
    NC_{N} - (N-1)C_{N-1} = 2N + 2C_{N-1}
\end{equation}\\
En divisant tout par $N(N-1)$:
\begin{equation}
    \frac{C_{N}}{N+1} = \frac{C_{N-1}}{N} + \frac{2}{N+1}
\end{equation}
En appliquant cette équation de manière répétée:
\begin{equation}
    \frac{C_{N}}{N+1} = \frac{C_{N-1}}{N} + \frac{2}{N+1}
\end{equation}
\begin{equation}
    \frac{C_{N}}{N+1} = \frac{C_{N-2}}{N-1} + \frac{2}{N} + \frac{2}{N+1}
\end{equation}
\begin{equation}
    \frac{C_{N}}{N+1} = \frac{C_{N-3}}{N-2} + \frac{2}{N-1} + \frac{2}{N} + \frac{2}{N+1}
\end{equation}
\begin{equation}
    \frac{C_{N}}{N+1} = \frac{2}{3} + \frac{2}{4} + \frac{2}{5} + \cdots + \frac{2}{N+1}
\end{equation}
En approximant cette somme par une intégrale:
\begin{equation}
    C_{N} = 2(N+1)(\frac{1}{3} + \frac{1}{4} + \frac{1}{5} + \cdots + \frac{1}{N+1})
\end{equation}
\begin{equation}
    C_{N} \approx \int_{3}^{N+1} \frac{1}{x} \,dx
\end{equation}
\begin{equation}
    C_{N} \approx 2(N+1)\ln N
\end{equation}
\begin{equation}
    C_{N} \approx 1.39N\log N
\end{equation}

\section{Axes d'optimisations}
On peut remplacer le deuxième appel récursif est une récursion terminale et peut être remplacée par une itération. 
Cela permet d'éviter une récurssion trop profonde dans le pire cas, le premier appel récursif traite le plus petit des deux partitions, le second est itéré.a 
\end{document}
